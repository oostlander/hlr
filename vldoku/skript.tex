%Schriftgröße, Layout, Papierformat, Art des Dokumentes
\documentclass[10pt,oneside,a4paper]{scrartcl}

%Einstellungen der Seitenränder
\usepackage[left=2.5cm,right=3.5cm,top=1.5cm,bottom=2.5cm,includeheadfoot]{geometry}

%neue Rechtschreibung
\usepackage[ngerman]{babel}

%Umlaute ermöglichen
\usepackage[utf8]{inputenc}

%Kopf- und Fußzeile
\usepackage{fancyhdr}
\pagestyle{fancy}
\fancyhf{}

%Linie oben
\renewcommand{\headrulewidth}{0.5pt}

%Linie unten
\renewcommand{\footrulewidth}{0.5pt}
\usepackage[pdfborder=0]{hyperref}

\author{Klaas P. Oostlander}

\title{Inoffizielles Skript zur Vorlesung Hochleistungsrechnen}

\begin{document}
\maketitle
\tableofcontents

\section{Einleitung}
\subsection{Organisatorisches}
Zur erfolgreichen Teilnahme an der Vorlesung und den begleitenden Übungen empfiehlt es sich, sich in die Mailingliste der Vorlesung einzutragen. Sie ist unter folgender url zu finden: \url{http://wr.informatik.uni-hamburg.de/listinfo/hr-1112}

Zunächst wollen wir uns die allgemeine Definition von Hochleistungsrechnen ansehen. Hierzu findet sich etwa in der Wikipedia:
\begin{description}
\item [Hochleistungsrechnen] (englisch: high-performance computing – HPC) ist ein Bereich des computergestützten Rechnens. Er umfasst alle Rechenarbeiten, deren Bearbeitung einer hohen Rechenleistung oder Speicherkapazität bedarf.
\item [Hochleistungsrechner] sind Rechnersysteme, die geeignet sind, Aufgaben des Hochleistungsrechnens zu bearbeiten.
\end{description}
Anwendungsbereiche sind in allen Wissenschaften mit einem hohen Bedarf an Rechen- und Speicherleistung.
Klassisch sind das:
\begin{itemize}
\item Physik (Astronomie, Teilchenphysik, …)
\item Erdsystemforschung (Klima, Ozeanographie, …)
\item Bioinformatik (Stammbaumberechnungen, Pharmazie, …)
\end{itemize}
Neu dazugekommen sind später:
\begin{itemize}
\item Finanzwirtschaft
\item Sozialwissenschaften (Simulation von Gesellschaften)
\end{itemize}
Generell lassen sich dabei die Rechneranlagen ihrer Größe nach unterteilen.
\begin{description}
\item[Klein]
  \begin{itemize}
  \item Mehrere Prozessoren in einem Rechner oder Prozessorkerne in einem Prozessor
  \item Einige hundert Euro
  \end{itemize}
\item[Groß]
  \begin{itemize}
  \item Hunderttausende von Prozessorkernen in einem Großrechner
  \item Plattenspeicher im Bereich einzelner Petabyte
  \item Bandarchive im Bereich dutzender Petabyte
  \item 10…200 Millionen Euro
  \end{itemize}
\end{description}
\section{Hardware- und Software-Konzepte}
\subsection{Hardware-Architekturen}

\subsection{Die TOP500-Liste}
Sehen sie sich die aktuelle TOP500-Liste an. Sie finden sie unter der Adresse  bla.
%TODO URL zur TOP500-Liste einfügen
\subsection{Vernetzungkonzepte}
\subsection{Hochleistungs-Eingabe/Ausgabe}
\subsection{Betriebssystemaspekte}
\section{Programmierung}
\section{Programmierwerkzeuge}
\section{Aktuelle Fragestellungen}

\end{document}
